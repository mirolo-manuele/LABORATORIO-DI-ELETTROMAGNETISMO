\documentclass[10pt,twocolumn]{article}
\usepackage[a4paper, left=1.5cm, right=1.5cm, top=2cm, bottom=3cm]{geometry}
\usepackage[T1]{fontenc}
\usepackage[utf8]{inputenc}
\usepackage[italian]{babel}
\usepackage{titling}
\usepackage{caption}
\usepackage{graphicx}
\usepackage{float}
\usepackage{relsize}
\usepackage{amsmath}
\usepackage{sectsty}
\usepackage{ragged2e}
\usepackage{booktabs}
\usepackage{physics}
\usepackage{xcolor}
\usepackage[most]{tcolorbox}
\usepackage{tikz-3dplot}
\usepackage{siunitx} % Carica prima siunitx
\usepackage[siunitx, american]{circuitikz} % Carica circuitikz UNA sola volta con le opzioni
\usepackage[colorlinks=true, linkcolor=black]{hyperref} 
\usepackage{enumitem} 

\setlist[itemize]{itemsep=4pt, parsep=1pt}

\newtcolorbox{nota}{
  blanker,
  before skip=1em,
  after skip=1em,
  left=1em,
  borderline west={1pt}{0pt}{black},
  fontupper=\itshape,
  before upper={\noindent\textbf{Nota}:\quad}
}



\begin{document}
\justifying
	\title{\textbf{Circuito raddrizzatore rettificatore}}
	\author{ \hspace{0.7cm} Ferrari Carola \hspace{0.7cm} Mirolo Manuele \hspace{0.7cm} Stroili Emanuele \hspace{0.7cm} Brusini Alessio}
	\date{29 Ottobre 2025}
	\maketitle
	\newgeometry{left=3cm, right=3cm, top=4cm, bottom=4cm}
	\onecolumn
\vspace{3cm}
	\begin{abstract}
		\centering
		\large
    L'esperimento consiste nella caratterizzazione di un circuito 
    rettificatore/raddrizzatore e nell'individuazione del valore del suo 
    coefficiente di ripple
       
	\end{abstract}

	\newpage
\restoregeometry
\twocolumn

\section{Apparato di misura}
 \begin{itemize}
    \item Trasformatore di tensione
    \item Generatore di tensione 
    \item Bread board
    \item Oscilloscopio 
    \item Condensatori elettrolici
    \item Resistenze
    \item Diodi 
 \end{itemize}

\section{Procedimento di misura}
In prima battuta si è proceduto con il costruire le curve 
volt-amperometriche dei quattro diodi che sono stati poi inseriti nel 
ponte di Graetz del circuito di cui si vogliono studiare le proprietà, 
questo per verificare che effettivamente avessero caratteristiche simili,
come da dichiarazione nominale. Successivamente sono stati analizzati
due diversi prototipi di circuiti raddrizzatori e rettificatori.
Per entrambi i circuiti sono state visualizzate sia la fase di raddrizzamento
che la fase di raddrizzamento e rettificazione, sono riportati di seguito 
i circuiti utilizzati:


\begin{figure}[H]
    \centering
\begin{circuitikz}[american]
    
    % Generatore di corrente continua variabile a sinistra
    \draw (0,0) to[rmeter, t=T] (0,4);
    
    % Resistenza in alto
    \draw (0,4) to[full diode] (3,4);

    % Induttanza
    \draw (3,4) to[R] (3,0);
    \draw (3,3) to (2,3);

    % Condensatore a destra
    \draw (2,3) to[C] (2,1);
    \draw (3,3) to (4,3);

    \draw (4,3) to[rmeter, t=osc] (4,1);
    \draw (4,1) to (3,1);
    \draw (2,1) to (3,1);
    \draw (3,0) to (0,0);
    %compila comunque, non so perché mi dia errore, se riuscite a risolvere meglio

\end{circuitikz}
\caption{Circuito utilizzato col diodo singolo.}
    \label{fig:diodo_singolo}
\end{figure}

\vspace{0.2 cm}
\begin{figure}[H]
    \centering    \includegraphics[width=0.4\textwidth]{Foto.pdf}
    \caption{Circuito con configurazione dei diodi a ponte di Gratez.}
    \label{fig:ponte_di_Graetz}
\end{figure}

\section{Grafici}

\begin{figure}[H]
    \centering    \includegraphics[width=0.4\textwidth]{codici/ponte/curva_diodi.pdf}
    \caption{Curve volt-amperometriche dei diodi utilizzati.}
    \label{fig:curva_diodi}
\end{figure}

Come detto in precedenza la prima operazione che abbiamo eseguito è stata quella di costruire le curve volt-amperometriche dei 
quattro diodi, in figura sono state riportate tutte insieme per rendere più semplice il confronto. Si osserva in modo 
evidente che il primo diodo ha una curva volt-amperometrica traslata verso destra rispetto agli altri tre diodi che possiedono curve sostanzialemente
equivalenti, questo mette in risalto una differenza dei parametri fisici che caratterizzano le giunzioni p-n che costituiscono i diodi,
 quali la propria resistenza interna o la tensione di soglia della giunzione.
\section{Coefficiente di ripple del circuito}

\section{Conclusione e commenti}

Confrontare efficienza ottenuta con i due diversi circuiti (dovrebbe venire circa doppia con ponte di Graetz)
Fare osservazioni su quale siano i valori migliori per raddrizzare e rettificatore

\end{document}
