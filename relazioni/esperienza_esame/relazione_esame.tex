\documentclass[10pt,twocolumn]{article}
\usepackage[a4paper, left=1.5cm, right=1.5cm, top=2cm, bottom=3cm]{geometry}
\usepackage[T1]{fontenc}
\usepackage[utf8]{inputenc}
\usepackage[italian]{babel}
\usepackage{titling}
\usepackage{caption}
\usepackage{graphicx}
\usepackage{float}
\usepackage{relsize}
\usepackage{amsmath}
\usepackage{sectsty}
\usepackage{ragged2e}
\usepackage{booktabs}
\usepackage{physics}
\usepackage{xcolor}
\usepackage[most]{tcolorbox}
\usepackage{tikz-3dplot}
\usepackage{siunitx} % Carica prima siunitx
\usepackage[siunitx, american]{circuitikz} % Carica circuitikz UNA sola volta con le opzioni
\usepackage[colorlinks=true, linkcolor=black]{hyperref} 
\usepackage{enumitem} 

\setlist[itemize]{itemsep=4pt, parsep=1pt}

\newtcolorbox{nota}{
  blanker,
  before skip=1em,
  after skip=1em,
  left=1em,
  borderline west={1pt}{0pt}{black},
  fontupper=\itshape,
  before upper={\noindent\textbf{Nota}:\quad}
}



\begin{document}
\justifying
	\title{\textbf{Circuito raddrizzatore rettificatore}}
	\author{ \hspace{0.7cm} Ferrari Carola \hspace{0.7cm} Mirolo Manuele \hspace{0.7cm} Stroili Emanuele \hspace{0.7cm} Brusini Alessio}
	\date{29 Ottobre 2025}
	\maketitle
	\newgeometry{left=3cm, right=3cm, top=4cm, bottom=4cm}
	\onecolumn
\vspace{3cm}
	\begin{abstract}
		\centering
		\large
    L'esperimento consiste nella caratterizzazione di un circuito 
    rettificatore/raddrizzatore e nell'individuazione del caratteristico 
    coefficiente di ripple.
       
	\end{abstract}

	\newpage
\restoregeometry
\twocolumn

\section{Strumentazione utilizzata}
 \begin{itemize}
    \item Trasformatore di tensione
    \item Generatore di tensione 
    \item Bread board
    \item Oscilloscopio 
    \item Condensatori elettrolici
    \item Resistenze
    \item Diodi 
 \end{itemize}

\vspace{-1cm}
\section{Procedimento di misura}
In prima battuta si è proceduto costruendo le curve 
volt-amperometriche dei quattro diodi, i quali sono stati successivamente inseriti nel 
ponte di Graetz del circuito di cui si vogliono studiare le proprietà, 
questo per verificare che effettivamente avessero caratteristiche simili,
come da dichiarazione nominale. Conseguentemente sono stati analizzati
due diversi prototipi di circuito raddrizzatore-rettificatore.
Per entrambi  sono state visualizzate sia la fase di raddrizzamento
che la fase di raddrizzamento e rettificazione, riportiamo di seguito una schematizzazione dei
 circuiti utilizzati:


\begin{figure}[H]
    \centering
\begin{circuitikz}[american]
    
    % Generatore di corrente continua variabile a sinistra
    \draw (0,0) to[rmeter, t=T] (0,4);
    
    % Resistenza in alto
    \draw (0,4) to[full diode] (3,4);

    % Induttanza
    \draw (3,4) to[R] (3,0);
    \draw (3,3) to (2,3);

    % Condensatore a destra
    \draw (2,3) to[C] (2,1);
    \draw (3,3) to (4,3);

    \draw (4,3) to[rmeter, t=osc] (4,1);
    \draw (4,1) to (3,1);
    \draw (2,1) to (3,1);
    \draw (3,0) to (0,0);
    %compila comunque, non so perché mi dia errore, se riuscite a risolvere meglio

\end{circuitikz}
\caption{Circuito utilizzato col diodo singolo.}
    \label{fig:diodo_singolo}
\end{figure}

\vspace{0.2 cm}
\begin{figure}[H]
    \centering    \includegraphics[width=0.52\textwidth]{Foto.pdf}
    \caption{Circuito con configurazione dei diodi a ponte di Gratez.}
    \label{fig:ponte_di_Graetz}
\end{figure}

\section{Grafici}
Ivi riportaimo il grafico che illustra le curve volt-amperometriche dei diodi ricavate sperimentalmente:
\begin{figure}[H]
    \centering    \includegraphics[width=0.52\textwidth]{codici/ponte/curva_diodi.pdf}
    \caption{Curve volt-amperometriche dei diodi utilizzati.}
    \label{fig:curva_diodi}
\end{figure}
 Si osserva in modo evidente che il primo diodo presenta una traslazione
 verso destra rispetto agli altri tre, che possiedono curve sostanzialemente
equivalenti. Ciò indica una probabile differenza dei parametri fisici che caratterizzano le giunzioni p-n costituenti i diodi,
 quali la propria resistenza interna o la tensione di soglia della giunzione.
 Si è tenuto in considerazione di quanto detto durante l'analisi dati.

 \begin{figure}[H]
    \centering    \includegraphics[width=0.52\textwidth]{codici/1 diodo/Grafici e parametri da mettere nella relazione/diodo_singolo_confronto_693.pdf}
    \caption{Grafico della tensione raddrizzata e rettificata nel circuito a diodo singolo con $R=693 \unit{\Omega}$.}
    \label{fig:curva_diodo_singolo}
\end{figure}

Il grafico evidenzia la differenza tra due diversi rapporti di RC.
Un $\tau$ non sufficientemente grande, \footnote[1]{si scelgono valori di R e C tali che $\tau >> \frac{T}{2}$, dove T è il periodo dell'onda analizzata}
infatti non permette di approssimare la curva di scarica del condensatore
con una costante. Il grafico sovrappone le curve del segnale alternato
in ingresso e il segnale raddrizzato e rettificato. E' importante evidenziare che il massimo voltaggio
del segnale rettificato è minore del massimo del segnale in ingresso. Spiegheremo il motivo di questo fenomeno in seguito.



 \begin{figure}[H]
    \centering    \includegraphics[width=0.52\textwidth]{codici/1 diodo/Grafici e parametri da mettere nella relazione/diodo_singolo_confronto_820.pdf}
    \caption{Grafico della tensione raddrizzata e rettificata nel circuito a diodo singolo con $R=820 \unit{\Omega}$.}
    \label{fig:curva_diodo_singolo2}
\end{figure}
In questo grafico si osserva il cosiddetto "segnale a orecchie di gatto", ovvero il
segnale raddrizzato dal circuito a singolo diodo, caratterizzato da dei semiperiodi
di segnale armonico (quando il diodo risulta polarizzato direttamente rispetto alla corrente) e 
da dei semiperiodi di segnale nullo (quando la corrente non fluisce nel diodi perché di verso opposto alla polarizzazione).\\

Riportiamo ora i grafici relativi al circuito con i diodi in configurazione a ponte di Graetz:


\begin{figure}[H]
    \centering    \includegraphics[width=0.52\textwidth]{codici/ponte/ponte_confronto.pdf}
    \caption{Grafico della tensione raddrizzata e rettificata nel circuito con configurazione dei diodi a ponte di Gratez con  $R=693 \unit{\Omega}\;e\;C=470 \unit{\micro\farad}$.}
    \label{fig:curva_ponte_confronto}
\end{figure}
In questo primo grafico vediamo il confronto tra segnale in ingresso, segnale raddrizzato 
e segnale rettificato. Notiamo nuovamente che con l'aggiunta di altri elementi circuitali diminuisce
il massimo raggiunto dalla tensione. Ciò è dato dalla resistenza propria del circuito, che
fa in modo che la differenza di potenziale ai capi della resistenza sia minore
di quella fornita dal generatore e dal processo di carica del condensatore, che non arriva
mai a completamento.

\begin{figure}[H]
    \centering    \includegraphics[width=0.52\textwidth]{codici/ponte/ponte_2.pdf}
    \caption{Grafico della tensione raddrizzata e rettificata nel circuito con configurazione dei diodi a ponte di Gratez con lo stesso valore di resistenza e due diversi valori di capacità}
    \label{fig:curva_ponte_2}
\end{figure}


Il grafico sopra riportato sovrappone due curve ottenute dal segnale raddrizzato tramite la
configurazione a ponte di Graetz, con due diversi valori di R e C (con R fissato e C variabile).
L'andamento della tensione in rosa dimostra come con un RC sufficientemente
piccolo si possa ottenere un voltaggio massimo pressoché uguale a quello del segnale originario, infatti $\tau$ 
è direttamente proporzionale al tempo di carica, ovvero ci vuole meno tempo per ottenere
una differenza di potenziale maggiore tra le due armature. D'altra parte questo implica anche
un processo di scarica più rapido (rispetto a una scelta di RC maggiore), e, consequenzialmente,
una peggior rettificazione del segnale.

\begin{figure}[H]
    \centering    \includegraphics[width=0.52\textwidth]{codici/ponte/ponte_invertito.pdf}
    \caption{Grafico della tensione raddrizzata e rettificata nel circuito con configurazione dei diodi a ponte di Gratez invertita }
    \label{fig:curva_ponte_invertito}
\end{figure}

È utile fare un confronto tra \ref{fig:curva_ponte_2} e \ref{fig:curva_ponte_invertito}, in quanto rappresentano
la medesima situazione con lo scambio di due diodi, al fine di verificare che
il circuito mostra pressoché lo stesso identico comportamento.

\begin{figure}[H]
    \centering    \includegraphics[width=0.4\textwidth]{codici/ponte/ponte_3.pdf}
    \caption{Grafico della tensione raddrizzata e rettificata nel circuito con configurazione dei diodi a ponte di Gratezcon  $R=693 \unit{\Omega}\;e\;C=470 \unit{\micro\farad}$.}
    \label{fig:curva_ponte_3}
\end{figure}


\section{Coefficiente di ripple del circuito}

\section{Conclusione e commenti}

Confrontare efficienza ottenuta con i due diversi circuiti (dovrebbe venire circa doppia con ponte di Graetz)
Fare osservazioni su quale siano i valori migliori per raddrizzare e rettificatore

\end{document}
