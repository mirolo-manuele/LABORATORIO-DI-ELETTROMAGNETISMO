\documentclass[10pt,twocolumn]{article}
\usepackage[a4paper, left=1.5cm, right=1.5cm, top=2cm, bottom=3cm]{geometry}
\usepackage[T1]{fontenc}
\usepackage[utf8]{inputenc}
\usepackage[italian]{babel}
\usepackage{titling}
\usepackage{caption}
\usepackage{graphicx}
\usepackage{float}
\usepackage{relsize}
\usepackage{amsmath}
\usepackage{sectsty}
\usepackage{ragged2e}
\usepackage{booktabs}
\usepackage{physics}
\usepackage{xcolor}
\usepackage[most]{tcolorbox}
\usepackage{tikz-3dplot}
\usepackage{siunitx} % Carica prima siunitx
\usepackage[siunitx, american]{circuitikz} % Carica circuitikz UNA sola volta con le opzioni
\usepackage[colorlinks=true, linkcolor=black]{hyperref} 
\usepackage{enumitem} 

\setlist[itemize]{itemsep=4pt, parsep=1pt}

\newtcolorbox{nota}{
  blanker,
  before skip=1em,
  after skip=1em,
  left=1em,
  borderline west={1pt}{0pt}{black},
  fontupper=\itshape,
  before upper={\noindent\textbf{Nota}:\quad}
}



\begin{document}
\justifying
	\title{\textbf{Circuito raddrizzatore rettificatore}}
	\author{ \hspace{0.7cm} Ferrari Carola \hspace{0.7cm} Mirolo Manuele \hspace{0.7cm} Stroili Emanuele \hspace{0.7cm} Brusini Alessio}
	\date{29 Ottobre 2025}
	\maketitle
	\newgeometry{left=3cm, right=3cm, top=4cm, bottom=4cm}
	\onecolumn
\vspace{3cm}
	\begin{abstract}
		\centering
		\large
    L'esperimento consiste nella caratterizzazione di un circuito 
    rettificatore/raddrizzatore e nell'individuazione del valore del suo 
    coefficiente di ripple
       
	\end{abstract}

	\newpage
\restoregeometry
\twocolumn

\section{Apparato di misura}
 \begin{itemize}
    \item Trasformatore di tensione
    \item Generatore di tensione 
    \item Bread board
    \item Oscilloscopio 
    \item Condensatori elettrolici
    \item Resistenze
    \item Diodi 
 \end{itemize}

\section{Procedimento di misura}
In prima battuta si è proceduto con il costruire le curve 
volt-amperometriche dei quattro diodi che sono stati poi inseriti nel 
ponte di Graetz del circuito di cui si vogliono studiare le proprietà, 
questo per verificare che effettivamente avessero caratteristiche simili,
come da dichiarazione nominale. Successivamente sono stati analizzati
due diversi prototipi di circuiti raddrizzatori e rettificatori.
Per entrambi i circuiti sono state visualizzate sia la fase di raddrizzamento
che la fase di raddrizzamento e rettificazione, sono riportati di seguito 
i circuiti utilizzati:


\begin{figure}[H]
    \centering
\begin{circuitikz}[american]
    
    % Generatore di corrente continua variabile a sinistra
    \draw (0,0) to[rmeter, t=T] (0,4);
    
    % Resistenza in alto
    \draw (0,4) to[full diode] (3,4);

    % Induttanza
    \draw (3,4) to[R] (3,0);
    \draw (3,3) to (2,3);

    % Condensatore a destra
    \draw (2,3) to[C] (2,1);
    \draw (3,3) to (4,3);

    \draw (4,3) to[rmeter, t=osc] (4,1);
    \draw (4,1) to (3,1);
    \draw (2,1) to (3,1);
    \draw (3,0) to (0,0);
    %compila comunque, non so perché mi dia errore, se riuscite a risolvere meglio

\end{circuitikz}
\caption{Circuito utilizzato col diodo singolo.}
    \label{fig:diodo_singolo}
\end{figure}

\vspace{0.2 cm}
\begin{figure}[H]
    \centering    \includegraphics[width=0.4\textwidth]{Foto.pdf}
    \caption{Circuito con configurazione dei diodi a ponte di Gratez.}
    \label{fig:ponte_di_Graetz}
\end{figure}

\section{Grafici}

\begin{figure}[H]
    \centering    \includegraphics[width=0.4\textwidth]{codici/ponte/curva_diodi.pdf}
    \caption{Curve volt-amperometriche dei diodi utilizzati.}
    \label{fig:curva_diodi}
\end{figure}

La prima operazione che abbiamo eseguita è stata quella di costruire le curve volt-amperometriche dei 
quattro diodi, riportate tutte nello stesso grafico per rendere più semplice il confronto. Si osserva in modo 
evidente che il primo diodo ha una curva volt-amperometrica traslata verso destra rispetto agli altri tre diodi che possiedono curve sostanzialemente
equivalenti, questo mette in risalto una differenza dei parametri fisici che caratterizzano le giunzioni p-n che costituiscono i diodi,
 quali la propria resistenza interna o la tensione di soglia della giunzione.

 \begin{figure}[H]
    \centering    \includegraphics[width=0.4\textwidth]{codici/1 diodo/Grafici e parametri da mettere nella relazione/diodo_singolo_confronto_693.pdf}
    \caption{Grafico della tensione raddrizzata e rettificata nel circuito a diodo singolo con $R=693 \unit{\Omega}$.}
    \label{fig:curva_diodo_singolo}
\end{figure}

Nel grafico riportato sopra si può notare la differenza tra due diversi rapporti di RC.
Un $\tau$ non sufficientemente grande %quantificare!!
infatti non permette di approssimare la curva di scarica del condensatore
con una costante. Si può inoltre notare il confronto tra il segnale alternato
in ingresso e il segnale raddrizzato e rettificato. Come ci si aspetta, il voltaggio
del segnale rettificato è minore del massimo del segnale in corrente alternata.



 \begin{figure}[H]
    \centering    \includegraphics[width=0.4\textwidth]{codici/1 diodo/Grafici e parametri da mettere nella relazione/diodo_singolo_confronto_820.pdf}
    \caption{Grafico della tensione raddrizzata e rettificata nel circuito a diodo singolo con $R=820 \unit{\Omega}$.}
    \label{fig:curva_diodo_singolo2}
\end{figure}
In questo grafico osserva il cosiddetto "segnale a orecchie di gatto", ovvero il
segnale raddrizzato dal circuito a singolo diodo, caratterizzato da dei semiperiodi
di segnale armonico (quando il diodo risulta polarizzato direttamente rispetto alla corrente) e 
da dei semiperiodi di segnale nullo (quando la corrente non fluisce nel diodi perché di verso opposto).

\begin{figure}[H]
    \centering    \includegraphics[width=0.4\textwidth]{codici/ponte/ponte_confronto.pdf}
    \caption{Grafico della tensione raddrizzata e rettificata nel circuito con configurazione dei diodi a ponte di Gratez con  $R=693 \unit{\Omega}\;e\;C=470 \unit{\micro\farad}$.}
    \label{fig:curva_ponte_confronto}
\end{figure}


\begin{figure}[H]
    \centering    \includegraphics[width=0.4\textwidth]{codici/ponte/ponte_2.pdf}
    \caption{Grafico della tensione raddrizzata e rettificata nel circuito con configurazione dei diodi a ponte di Gratez con lo stesso valore di resistenza e due diversi valori di capacità}
    \label{fig:curva_ponte_2}
\end{figure}



\begin{figure}[H]
    \centering    \includegraphics[width=0.4\textwidth]{codici/ponte/ponte_3.pdf}
    \caption{Grafico della tensione raddrizzata e rettificata nel circuito con configurazione dei diodi a ponte di Gratezcon  $R=693 \unit{\Omega} e C=470 \unit{\micro\farad}$.}
    \label{fig:curva_ponte_3}
\end{figure}

\begin{figure}[H]
    \centering    \includegraphics[width=0.4\textwidth]{codici/ponte/ponte_invertito.pdf}
    \caption{Grafico della tensione raddrizzata e rettificata nel circuito con configurazione dei diodi a ponte di Gratez invertita }
    \label{fig:curva_ponte_invertito}
\end{figure}

Le figure \ref{fig:curva_diodo_singolo}, \ref{fig:curva_diodo_singolo2} rappresentano il grafico della tensione d'ingresso ($V_in$) confrontato
con i due grafici della tensione a capi della resistenza (circuito raddrizzato) e della tensione con l'aggiunta del condensatore condensatore
una data capacità( circuito raddrizzato e rettificato). Le figure \ref{fig:curva_ponte_confronto}, \ref{fig:curva_ponte_2}, \ref{fig:curva_ponte_3} invece mettono a confronto 
gli stessi tipi di grafici ma con il circuito con i diodi in configurazione a ponte di Gratez, infine in figura \ref{fig:curva_ponte_invertito}
 è rappresentato il caso in cui si sono scambiate (...manca come sono stai scambiati i diodi) le posizioni dei diodi per verificare se
 ci potessero essere differenze col caso precedente. Una caratteristica comune a tutti  grafici è che la tensione raddrizzata e rettificata risulta
 avere dei massimi con valore minore rispetto a quelli del grafico della tensione in entrata senza la resistenza e la capacità. questo 
 è dovuto dal fatto che una volta superata la tensione di soglia all'interno dei diodi è presente una differenza di potenziale tale 
 da permettere il passaggio di cariche nelle proprie giunzioni p-n, questa andrà ad influire sulla tensone ai capi della resistenza 
 diminuendone i massimi. 

\section{Coefficiente di ripple del circuito}

\section{Conclusione e commenti}

Confrontare efficienza ottenuta con i due diversi circuiti (dovrebbe venire circa doppia con ponte di Graetz)
Fare osservazioni su quale siano i valori migliori per raddrizzare e rettificatore

\end{document}
