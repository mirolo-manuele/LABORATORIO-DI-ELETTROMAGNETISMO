\documentclass[10pt,twocolumn]{article}
\usepackage[a4paper, left=1.5cm, right=1.5cm, top=2cm, bottom=3cm]{geometry}
\usepackage[T1]{fontenc}
\usepackage[utf8]{inputenc}
\usepackage[italian]{babel}
\usepackage{amsmath}
\usepackage{titling}
\usepackage{caption}
\usepackage{graphicx}
\usepackage{float}
\usepackage{relsize}
\usepackage{amsmath}
\usepackage{sectsty}
\usepackage{ragged2e}
\usepackage{circuitikz}
\usepackage{booktabs}
\usepackage{enumitem}
\usepackage{tikz}
\usepackage{physics}
\usepackage{xcolor}
\usepackage[most]{tcolorbox}
\usepackage{tikz-3dplot}
\usepackage{tikz}
\usepackage{ragged2e}
\usepackage{siunitx}
% \usepackage{booktabs}
\usepackage[colorlinks=true, linkcolor=black]{hyperref}  %per rendere l'indice genrale "interattivo"
\usepackage{enumitem}  %distanza degli itemize
\setlist[itemize]{itemsep=4pt, parsep=1pt}
\newtcolorbox{nota}{
  blanker,
  before skip=1em,
  after skip=1em,
  left=1em,
  borderline west={1pt}{0pt}{black},
  fontupper=\itshape,
  before upper={\noindent\textbf{Nota}:\quad}
}

\begin{document}
\justifying
	\title{\textbf{Esperienze circuiti RC e RLC}}
	\author{Ferrari Carola \hspace{0.7cm} Mirolo Manuele \hspace{0.7cm} Stroili Emanuele}
	\date{14/12/2025}
	\maketitle
	\newgeometry{left=3cm, right=3cm, top=4cm, bottom=4cm}
	\onecolumn
	\tableofcontents
\vspace{3cm}
	\begin{abstract}
		\centering
		\large
    L'esperimento consiste nello studio di due circuiti, uno RC ed uno RLC, stimolati da onde quadre allo scopo di visualizzare l'oscillazione smorzata. Inoltre i due vengono stimolati da onde sinusoidali per osservare l'effetto di risonanza.
    %La parte sulla lampadina al neon va messa nella relazione?? No
       
	\end{abstract}

	\newpage
\restoregeometry
\twocolumn

\section{Apparato sperimentale}
\subsection{Circuito RC}
\begin{center}
\begin{circuitikz}[american]
    
    % Generatore di corrente continua variabile a sinistra
    \draw (0,0) to[rmeter, t=G] (0,3);
    
    % Resistenza in alto
    \draw (0,3) to[R, l=$R$] (3,3);
    
    % Condensatore a destra
    \draw (3,3) to[C, l=$C$] (3,0);
    
    % Amperometro in basso
    \draw (0,0) to (3,0);
    
    \draw (3,2) to[short] (2,2) 
    % Voltmetro in parallelo alla lampadina
    to[rmeter, t=osc] (2,1)
    to[short] (3,1);
\end{circuitikz}
\end{center}


\subsection{Circuito RLC}

\begin{center}
\begin{circuitikz}[american]
    
    % Generatore di corrente continua variabile a sinistra
    \draw (0,0) to[rmeter, t=G] (0,5);
    
    % Resistenza in alto
    \draw (0,4) to[R, l=$R$] (3,4);

    % Induttanza
    \draw (3,4) to[L, l=$L$] (3,1);

    % Condensatore a destra
    \draw (3,1) to[C, l=$C$] (0,1);

    \draw (0,0) to (0,1);
    \draw (0,4) to (0,5);
    \draw (0,5) to (4,5);
    \draw (4,5) to[rmeter, t=osc] (4,0);
    \draw (4,0) to (0,0);
    \draw (3,4) to (4,4);
    \draw (3,1) to (4,1);
    %compila comunque, non so perché mi dia errore, se riuscite a risolvere meglio

\end{circuitikz}
\end{center}

\section{Procedimento di misura}
Si sono usati i seguenti strumenti:
\begin{itemize}
  \item Oscilloscopio digitale
  \item Generatore di segnali periodici
  \item Cavi conduttori
  \item Resistenza variabile
  \item Capacità variabile
  \item Induttanza variabile
\end{itemize}
\subsection{Circuito RC}
Per visualizzare i processi di carica e scarica del condensatore si è 
costruito un circuito RC con l'oscilloscopio in parallelo al condensatore. Per poter valutare correttamente il
fenomeno è necessario scegliere un'opportuna frequenza di onda quadra. Si può considerare buono il segnale visualizzato
sull'oscilloscopio quando la parte finale della curva di carica/scarica sembra avere tangente orizzontale. Infatti, non è possibile 
 ottenere una precisione migliore a causa delle interferenze dell'ambiente, dalla quale non si può prescindere.
\subsection{Circuito RLC}
\subsubsection{Oscillazione smorzata}
Come per il caso del circuito RC è necessario impostare l'onda quadra a una frequenza tale da non essere più in grado
di percepire la differenza tra interferenza e oscillazione smorzata nella parte finale del segnale visualizzato
sull'oscilloscopio.
\subsubsection{Risonanza}
Per ricostruire la Lorentziana tipica del grafico ottenuto analizzando la risonanza è necessario cambiare molto lentamente la frequenza dell'onda sinusoidale
dal generatore al fine di individuare, osservando l'oscilloscopio, l'intervallo in cui l'ampiezza del segnale aumenta.

\section{Grafici}
\subsection{Circuito RC}

In seguito alla risoluzione dell'equazione differenziale del circuito è possibile ricavare la carica accumulata nelle lamine del condensatore e la corrente passante all'interno del circuito in funzione del tempo. 
Da quest'ultima è possibile passare all'espressione della dellogaritmo naturale della tensione a capi del condensatore in funzione del tempo, il cui andamento è dato da:
\[
ln(V(t))=ln(V_0)- \frac{1}{RC}t
\]
\begin{figure}[H]
    \centering
    \includegraphics[width=0.5\textwidth]{presa_dati_manuale.pdf}
    \caption{andamento della tensione a capi del capacitore}
    \label{fig:esempio}
\end{figure}



\begin{figure}[h]
    \centering
    \includegraphics[width=0.5\textwidth]{fit_presa_dati_manuale.pdf}
    \caption{grafico in scala logaritmica}
    \label{fig:esempio}
\end{figure}
 
Questi due grafici rappresentano l'andamento della tensione nella sua forma esponenizale e nella sua espressione in scala logaritmica. I dati sono stati presi manualemnte per osservare e verificare che effettivmente l''andamento sperimentale rispecchi almeno in parte quello teorico.
È possibile osservare sia dal primo ma soprattutto dal secondo grafico che nella fase di carica il si riscontra l'effettivo andamento lineare mentre dagli istanti di tempo successivi a $ 6 \cdot 10^{-6} s$ risulta esserci una deviazione dalla linearità questo probabilmente è  causato dal raggiungimento della tensione di regime, dove il tasso di variazione di quest'ultima si avvicina a zero:
la variazione fisica esponenziale è così piccola che viene sopraffatta dal rumore di fondo e dai limiti di risoluzione dello strumento, distorcendo la pendenza della retta logaritmica.


\subsection{Circuito RLC}
\subsubsection{Oscillazione smorzata}
\begin{figure}[H] % [h] = here, posizione approssimativa
  \centering
  \includegraphics[width=0.5\textwidth]{oscillazione_smorzata_2.pdf} % o .png, .pdf, ecc.
  \label{fig:I/Oscillazione smorzata}
\end{figure}
Nel grafico si vede l'andamendo dell'oscillazione smorzata
del circuito RLC, che è descritto dall'equazione: 
\[V(t)=V_0e^{-\Gamma t/2}cos(\omega_0t+\phi)\]
dove $\Gamma=\frac{R}{L}$ e $\omega_0=\frac{1}{\sqrt{LC}}$.
Note R e C si può stimare il valore di L in due modi:
facendo un fit semilogaritmico dell'andamento dei massimi e
valutando il periodo dell'oscillazione smorzata.

\begin{figure}[H] % [h] = here, posizione approssimativa
  \centering
  \includegraphics[width=0.5\textwidth]{curva_esponenizale_2.pdf} % o .png, .pdf, ecc.
  \label{fig:I/Curva esponenziale}
\end{figure}

Nel grafico si può notare l'andamento dei massimi
e dei minimi (presi in valore assoluto). La forma risulta abbastanza
simile a un'esponenziale, le inconguenze rispetto al risultato
atteso sono da attribuire al rumore di rete e a una possibile
imprecisione sulla taratura dello zero nell'asse delle tensioni,
dovuta anche questa all'incertezza data dalle interferenze sulla 
misura.

\begin{figure}[H] % [h] = here, posizione approssimativa
  \centering
  \includegraphics[width=0.5\textwidth]{semilog_2.pdf} % o .png, .pdf, ecc.
  \label{fig:I/Fit semilogaritmico}
\end{figure}

Graficando la curva esponenziale vista nel grafico precedente
in scala semilogaritmica, si può fare un fit lineare per ottenere
una stima di $-\frac{\Gamma}{2}$ e conseguentemente una stima di L.
Eseguendo i calcoli si ottiene $L_{\Gamma_1}=(32,7\pm2,7)mH$.

A questo punto e possibile, tramite i periodi misurati stimare il valore di L
direttamente dalla frequenza usando: $L=\frac{1}{\omega_0^2 C}$, da cui si ottiene $L_{\omega_1}=(26,4\pm4,8)mH$

\subsubsection{Risonanza}
\begin{figure}[H] % [h] = here, posizione approssimativa
  \centering
  \includegraphics[width=0.5\textwidth]{risonanza.pdf} % o .png, .pdf, ecc.
  \label{fig:I/Risonanza}
\end{figure}
Per stimare $\Gamma$ a partire dall'effetto di risonanza, bisogna
valutare l'ampiezza della Lorentziana a $\frac{V_0Q_s}{\sqrt{2}}$,
dove $V_0Q_s$ è il valore del massimo della curva, che viene trovato
tramite un'approssimazione al secondo ordine, ovvero ricavando
l'equazione della parabola passante per i tre punti con valore di
tensione maggiore e prendendo il suo vertica come massimo della
Lorentziana. L'ampiezza viene valutata con un'approssimazione al
primo ordine tra i due punti immediatamente sopra e sotto alla retta
$\frac{V_0Q_s}{\sqrt{2}}$. In questo modo si ottiene $L_{\Gamma_2}=18.81mH$ ed $L_{\omega_2}=26.39mH$
%non ho ben capito come è ottenuto L_omega nel file Esperienza_5\risonanza e perché questi valori sono così sbagliati7


\section{Risultati}
\subsection{Circuito RC}
Si è creato un codice python in modo tale da iterare il fit lineare su 100 estremi diversi dell'intervallo temporale.
La tabella qui riportata elenca i valori finali con $\chi^2$ minore, ovverosia quelli ricavati col set che rappresenta meglio il modello:
\begin{table}[H]
\centering
\begin{tabular}{|c|c|c|c|}
\hline
RC ottenuto & $\Delta $ & $\chi^2$ & RC atteso  \\ \hline
$2.01\times 10^{-5}$ &  $7.8\times 10^{-8}$ & $0.02$ & $1.61\times 10^{-6} $\\ \hline
\end{tabular}
\caption*{Prova 1}
\end{table}
\begin{table}[H]
\centering
\begin{tabular}{|c|c|c|c|}
\hline
RC ottenuto & $\Delta $ & $\chi^2$ & RC atteso  \\ \hline
$4.16\times 10^{-6}$ &  $5.4\times 10^{-8}$ & $0.02$ & $2.4\times 10^{-6} $\\ \hline
\end{tabular}
\caption*{Prova 2}
\end{table}
\subsection{Circuito RLC}
Ivi riportiamo i risultati ottenuti per il circuito RLC\footnote[1]{Non vengono riportati gli errori su $L_{\Gamma_2}$ e $L_{\omega_2}$ in quanto ricavati dall'espressione della parabola passante per tre punti, esente da stima di errore.}:
\begin{table}[H]
\centering
\begin{tabular}{|c|c|c|}
\hline
L & Valore $\text{mH}$ & $\Delta \text{ mH}$ \\ \hline
$L_{\Gamma_1}$ &  $32,7 $ & $2,7 $ \\ \hline
$L_{\omega_1}$ &  $26,4 $ & $ 4,8 $ \\ \hline
$L_{\Gamma_2}$ &  $18,81 $ & N/D \\ \hline
$L_{\omega_2}$ &  $26,39 $ & N/D \\ \hline
\end{tabular}

\end{table}
I valori ottenuti per L sono $L=(32,7\pm2,7)mH$ con il fit in scala semilogaritmica e $L=(26,4\pm4,8)mH$
dalla misura del periodo dell'oscillazione smorzata, il valore di induttanza che era stato impostato
era $L=42mH$.

\section{Conclusioni}
I valori ottenuti risultano non comatibili, nè fra di loro nè con il valore impostato, l'errore atteso maggiore era su $L_w$,
in quanto tale metodo dipendeva fortemente dall'interpolazione svoltasi durante la presa dati. \\
Tale ipotesi risulta confermata dai dati ottenuti. \\
Tuttavia, ci si attendeva un errore massimo del $15\%$ sul valore di L. Cause probabili di tale discrepanze sono un errore sistematico
nella determinazione dello zero dell'oscilloscopio (più specificatamente una sottostima di tale valore)
\end{document}

