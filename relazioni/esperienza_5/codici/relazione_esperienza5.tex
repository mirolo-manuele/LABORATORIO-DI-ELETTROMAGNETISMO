\documentclass[10pt,twocolumn]{article}
\usepackage[a4paper, left=1.5cm, right=1.5cm, top=2cm, bottom=3cm]{geometry}
\usepackage[T1]{fontenc}
\usepackage[utf8]{inputenc}
\usepackage[italian]{babel}
\usepackage{amsmath}
\usepackage{titling}
\usepackage{caption}
\usepackage{graphicx}
\usepackage{float}
\usepackage{relsize}
\usepackage{amsmath}
\usepackage{sectsty}
\usepackage{ragged2e}
\usepackage{circuitikz}
\usepackage{booktabs}
\usepackage{enumitem}
\usepackage{tikz}
\usepackage{physics}
\usepackage{xcolor}
\usepackage[most]{tcolorbox}
\usepackage{tikz-3dplot}
\usepackage{tikz}
\usepackage{ragged2e}
\usepackage{siunitx}
% \usepackage{booktabs}
\usepackage[colorlinks=true, linkcolor=black]{hyperref}  %per rendere l'indice genrale "interattivo"
\usepackage{enumitem}  %distanza degli itemize
\setlist[itemize]{itemsep=4pt, parsep=1pt}
\newtcolorbox{nota}{
  blanker,
  before skip=1em,
  after skip=1em,
  left=1em,
  borderline west={1pt}{0pt}{black},
  fontupper=\itshape,
  before upper={\noindent\textbf{Nota}:\quad}
}

\begin{document}
\justifying
	\title{\textbf{Misura della costante di Faraday}}
	\author{Ferrari Carola \hspace{0.7cm} Mirolo Manuele \hspace{0.7cm} Stroili Emanuele}
	\date{18 Novembre 2025}
	\maketitle
	\newgeometry{left=3cm, right=3cm, top=4cm, bottom=4cm}
	\onecolumn
	\tableofcontents
\vspace{3cm}
	\begin{abstract}
		\centering
		\large
    L'esperimento consiste nello studio di un circuito RC stimolato da onde quadre e di un circuito RLC stimolato da onde quadre allo scopo di visualizzare l'oscillazione smorzata e da onde sinusoidali per osservare l'effetto di risonanza.
    %La parte sulla lampadina al neon va messa nella relazione??
       
	\end{abstract}

	\newpage
\restoregeometry
\twocolumn

\section{Apparato sperimentale}
\subsection{Circuito RC}
\begin{center}
\begin{circuitikz}[american]
    
    % Generatore di corrente continua variabile a sinistra
    \draw (0,0) to[rmeter, t=G] (0,3);
    
    % Resistenza in alto
    \draw (0,3) to[R, l=$R$] (3,3);
    
    % Condensatore a destra
    \draw (3,3) to[C, l=$C$] (3,0);
    
    % Amperometro in basso
    \draw (0,0) to (3,0);
    
    \draw (3,2) to[short] (2,2) 
    % Voltmetro in parallelo alla lampadina
    to[rmeter, t=osc] (2,1)
    to[short] (3,1);
\end{circuitikz}
\end{center}


\subsection{Circuito RLC}

\begin{center}
\begin{circuitikz}[american]
    
    % Generatore di corrente continua variabile a sinistra
    \draw (0,0) to[rmeter, t=G] (0,5);
    
    % Resistenza in alto
    \draw (0,4) to[R, l=$R$] (3,4);

    % Induttanza
    \draw (3,4) to[L, l=$L$] (3,1);

    % Condensatore a destra
    \draw (3,1) to[C, l=$C$] (0,1);

    \draw (0,0) to (0,1)
    \draw (0,4) to (0,5)
    \draw (0,5) to (4,5)
    \draw (4,5) to[rmeter, t=osc] (4,0)
    \draw (4,0) to (0,0)
    \draw (3,4) to (4,4)
    \draw (3,1) to (4,1)
    %compila comunque, non so perché mi dia errore, se riuscite a risolvere meglio

\end{circuitikz}
\end{center}

\section{Procedimento di misura}
Si sono usati i seguenti strumenti:
\begin{itemize}
  \item Oscilloscopio digitale
  \item Generatore di segnali periodici
  \item Cavi conduttori
  \item Resistenza variabile
  \item Capacità variabile
  \item Induttanza variabile
\end{itemize}
\subsection{Circuito RC}
Per visualizzare il cosiddetto il processo di carica e scarica del condensatore si è 
costruito un circuito RC con l'oscilloscopio in parallelo al condensatore. Per poter valutare correttamente il
fenomeno è necessario scegliere un'opportuna frequenza di onda quadra. Si può considerare buono il segnale visualizzato
sull'oscilloscopio quando la parte finale della curva di carica/scarica sembra avere tangente orizzontale. Non sarebbe possibile 
infatti ottenere una precisione migliore a causa delle interferenze dell'ambiente, da cui non si può prescindere.
\subsection{Circuito RLC}
\subsubsection{Oscillazione smorzata}
Come per il caso del circuito RC è necessario impostare l'onda quadra a una frequenza tale da non essere più in grado
di percepire la differenza tra interferenza e oscillazione smorzata nella parte finale del segnale visualizzato
sull'oscilloscopio.
\subsubsection{Risonanza}
Per ricostruire la Laurentiana della risonanza è necessario cambiare molto lentamente la frequenza dell'onda sinusoidale
dal generatore al fine di individuare, osservando l'oscilloscopio, l'intervallo in cui l'ampiezza del segnale aumenta.

\section{Grafici}
\subsection{Circuito RC}

\subsection{Circuito RLC}
\begin{figure}[H] % [h] = here, posizione approssimativa
  \centering
  \includegraphics[width=0.5\textwidth]{oscillazione_smorzata_2.pdf} % o .png, .pdf, ecc.
  \label{fig:I/Oscillazione smorzata}
\end{figure}
Nel grafico si vede l'andamendo dell'oscillazione smorzata
del circuito RLC, che è descritto dall'equazione: 
\[V(t)=V_0e^{-\Gamma t/2}cos(\omega_0t+\phi)\]
dove $\Gamma=\frac{R}{L}$ e $\omega_0=\frac{1}{\sqrt{LC}}$.
Note R e C si può stimare il valore di L in due modi:
facendo un fit semilogaritmico dell'andamento dei massimi e
valutando il periodo dell'oscillazione smorzata.

\begin{figure}[H] % [h] = here, posizione approssimativa
  \centering
  \includegraphics[width=0.5\textwidth]{curva_esponenizale_2.pdf} % o .png, .pdf, ecc.
  \label{fig:I/Curva esponenziale}
\end{figure}

Nel grafico si può notare l'andamento dei massimi
e dei minimi (presi in valore assoluto). La forma risulta abbastanza
simile a un'esponenziale, le inconguenze rispetto al risultato
atteso sono da attribuire al rumore di rete e a una possibile
imprecisione sulla taratura dello zero nell'asse delle tensioni,
dovuta anche questa all'incertezza data dalle interferenze sulla 
misura.

\begin{figure}[H] % [h] = here, posizione approssimativa
  \centering
  \includegraphics[width=0.5\textwidth]{semilog_2.pdf} % o .png, .pdf, ecc.
  \label{fig:I/Fit semilogaritmico}
\end{figure}

Graficando la curva esponenziale vista nel grafico precedente
in scala semilogaritmica, si può fare un fit lineare per ottenere
una stima di $-\frac{\Omega}{2}$ e conseguentemente una stima di L.
Eseguendo i calcoli si ottiene $L=(32,7\pm2,7)mH$.

\section{Risultati}
\subsection{Circuito RC}

\subsection{Circuito RLC}
I valori ottenuti per L sono $L=(32,7\pm2,7)mH$ con il fit in scala semilogaritmica e $L=(26,4\pm4,8)mH$
dalla misura del periodo dell'oscillazione smorzata, il valore di induttanza che era stato impostato
era $L=42mH$.

\section{Conclusioni}

\end{document}