\documentclass[10pt,twocolumn]{article}
\usepackage[a4paper, left=1.5cm, right=1.5cm, top=2cm, bottom=2cm]{geometry}
\usepackage[T1]{fontenc}
\usepackage[utf8]{inputenc}
\usepackage[italian]{babel}
\usepackage{amsmath}
\usepackage{titling}
\usepackage{caption}
\usepackage{graphicx}
\usepackage{float}
\usepackage{relsize}
\usepackage{amsmath}
\usepackage{sectsty}
\usepackage{ragged2e}
\usepackage{circuitikz}

\usepackage{booktabs}
\usepackage{enumitem}
\usepackage{tikz}
\usepackage{physics}
\usepackage{xcolor}
\usepackage[most]{tcolorbox}
\usepackage{tikz-3dplot}
\usepackage{tikz}
\usepackage{ragged2e}
\usepackage{siunitx}
% \usepackage{booktabs}
\usepackage[colorlinks=true, linkcolor=black]{hyperref}  %per rendere l'indice genrale "interattivo"
\usepackage{enumitem}  %distanza degli itemize
\setlist[itemize]{itemsep=4pt, parsep=1pt}
\newtcolorbox{nota}{
  blanker,
  before skip=1em,
  after skip=1em,
  left=1em,
  borderline west={1pt}{0pt}{black},
  fontupper=\itshape,
  before upper={\noindent\textbf{Nota}:\quad}
}



\begin{document}
\justifying
	\title{\textbf{Misura del calore specifico solidi e calore latente di fusione del ghiaccio}}
	\author{Brusini Alessio \hspace{0.7cm} Ferrari Carola \hspace{0.7cm} Mirolo Manuele \hspace{0.7cm} Stroili Emanuele}
	\date{14 Ottobre 2025}
	\maketitle
	\newgeometry{left=3cm, right=3cm, top=4cm, bottom=4cm}
	\onecolumn
	\tableofcontents
\vspace{3cm}
	\begin{abstract}
		\centering
		\large
    L'esperimento consiste nell'ottenere la curva volt-amperometrica di una 
    lampadina a filamento, partendo da tensioni basse fino alla fusione del 
    tungsteno. L'obiettivo è verificare l'andamento non ohmico della resistenza
    interna della lampadina, 
       
	\end{abstract}

	\newpage
\restoregeometry
\twocolumn

\section{Apparato sperimentale}
\subsection{Misura per bassi voltaggi}
Per ottenere la misura a bassi voltaggi si costruisce un circuito composto da:
\begin{itemize}
    \item Generatore di corrente continua
    \item Resistenza 
    \item Voltmetro
    \item Amperometro
    \item Fotodiodo
    \item Lampadina con tensione di funzionamento 6V
\end{itemize}
\begin{center}
\begin{circuitikz}[american]
    
    % Generatore di corrente continua variabile a sinistra
    \draw (0,0) to[battery1, invert, l=$V$] (0,3);
    
    % Resistenza in alto
    \draw (0,3) to[R, l=$R$] (3,3);
    
    % Lampadina a destra
    \draw (3,3) to[lamp, l=$L$] (3,0);
    
    % Amperometro in basso
    \draw (0,0) to[rmeter, t=A] (3,0);
    
    \draw (3,2) to[short] (4.5,2) 
    % Voltmetro in parallelo alla lampadina
    to[rmeter, t=V] (4.5,1)
    to[short] (3,1);
\end{circuitikz}
\end{center}

\section{Procedimento di misura}
La misura si svolge in due fasi: nella prima si prendono misure più fitte
per poter apprezzare le oscillazioni di corrente. A questo scopo è necessario
introdurre una resistenza nel circuito da utilizzare come partitore di tensione.
Inoltre, nella prima fase, ci si serve di un fotodiodo per poter captare la 
flebile luminescenza della lampadina, non visibile univocamente a occhio nudo.\\
Nella seconda fase si prendono dati meno fitti, perciò si rimuovono la 
resistenza e il fotodiodo dal circuito, non più necessari nella misura.

\section{Dati}
\begin{table}[H]
    \begin{minipage}{0.5\textwidth}

\centering
\caption*{}
\label{tab:temp2}
\begin{tabular}{|r|r|}
\hline
Tensione(V) & Corrente (A) \\ \hline
0 & 17.2 \\ \hline
30 & 17.2 \\ \hline
60 & 17.2 \\ \hline
90 & 17.2 \\ \hline
120 & 17.1 \\ \hline
150 & 17.1 \\ \hline
180 & 17.1 \\ \hline
210 & 17.2 \\ \hline
240 & 17.1 \\ \hline
270 & 17.1 \\ \hline
300 & 17.2 \\ \hline
330 & 17.2 \\ \hline
360 & 17.2 \\ \hline
390 & 36.9 \\ \hline
420 & 36.8 \\ \hline
450 & 36.7 \\ \hline
480 & 36.6 \\ \hline
510 & 36.5 \\ \hline
540 & 36.4 \\ \hline
570 & 36.4 \\ \hline
600 & 36.3 \\ \hline
630 & 36.1 \\ \hline
660 & 36.1 \\ \hline
690 & 36.0 \\ \hline
720 & 35.9 \\ \hline
750 & 35.9 \\ \hline
780 & 35.8 \\ \hline
810 & 35.8 \\ \hline
840 & 35.7 \\ \hline
870 & 35.7 \\ \hline
900 & 35.7 \\ \hline
930 & 35.7 \\ \hline
960 & 35.6 \\ \hline
990 & 35.6 \\ \hline
1020 & 35.5 \\ \hline
    \end{tabular}
    \end{minipage}
\hfill
\begin{minipage}{0.5\textwidth}

\section{Analisi dati}
\section{Grafici}
\section{Conclusione}
%Scegliere il multimetro piccolino!!

\end{document}















































\end{document}