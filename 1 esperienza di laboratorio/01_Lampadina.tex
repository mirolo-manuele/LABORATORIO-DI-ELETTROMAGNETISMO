\documentclass[10pt,twocolumn]{article}
\usepackage[a4paper, left=1.5cm, right=1.5cm, top=2cm, bottom=2cm]{geometry}
\usepackage[T1]{fontenc}
\usepackage[utf8]{inputenc}
\usepackage[italian]{babel}
\usepackage{amsmath}
\usepackage{titling}
\usepackage{caption}
\usepackage{graphicx}
\usepackage{float}
\usepackage{relsize}
\usepackage{amsmath}
\usepackage{sectsty}
\usepackage{ragged2e}
\usepackage{circuitikz}

\usepackage{booktabs}
\usepackage{enumitem}
\usepackage{tikz}
\usepackage{physics}
\usepackage{xcolor}
\usepackage[most]{tcolorbox}
\usepackage{tikz-3dplot}
\usepackage{tikz}
\usepackage{ragged2e}
\usepackage{siunitx}
% \usepackage{booktabs}
\usepackage[colorlinks=true, linkcolor=black]{hyperref}  %per rendere l'indice genrale "interattivo"
\usepackage{enumitem}  %distanza degli itemize
\setlist[itemize]{itemsep=4pt, parsep=1pt}
\newtcolorbox{nota}{
  blanker,
  before skip=1em,
  after skip=1em,
  left=1em,
  borderline west={1pt}{0pt}{black},
  fontupper=\itshape,
  before upper={\noindent\textbf{Nota}:\quad}
}



\begin{document}
\justifying
	\title{\textbf{Misura del calore specifico solidi e calore latente di fusione del ghiaccio}}
	\author{Brusini Alessio \hspace{0.7cm} Ferrari Carola \hspace{0.7cm} Mirolo Manuele \hspace{0.7cm} Stroili Emanuele}
	\date{14 Ottobre 2025}
	\maketitle
	\newgeometry{left=3cm, right=3cm, top=4cm, bottom=4cm}
	\onecolumn
	\tableofcontents
\vspace{3cm}
	\begin{abstract}
		\centering
		\large
    L'esperimento consiste nell'ottenere la curva volt-amperometrica di una 
    lampadina a filamento, partendo da tensioni basse fino alla fusione del 
    tungsteno. L'obiettivo è verificare l'andamento non ohmico della resistenza
    interna della lampadina, 
       
	\end{abstract}

	\newpage
\restoregeometry
\twocolumn

\section{Apparato sperimentale}
\subsection{Misura per bassi voltaggi}
Per ottenere la misura a bassi voltaggi si costruisce un circuito composto da:
\begin{itemize}
    \item Generatore di corrente continua
    \item Resistenza 
    \item Voltmetro
    \item Amperometro
    \item Fotodiodo
    \item Lampadina con tensione di funzionamento 6V
\end{itemize}
\begin{center}
\begin{circuitikz}[american]
    
    % Generatore di corrente continua variabile a sinistra
    \draw (0,0) to[battery1, invert, l=$V$] (0,3);
    
    % Resistenza in alto
    \draw (0,3) to[R, l=$R$] (3,3);
    
    % Lampadina a destra
    \draw (3,3) to[lamp, l=$L$] (3,0);
    
    % Amperometro in basso
    \draw (0,0) to[rmeter, t=A] (3,0);
    
    \draw (3,2) to[short] (4.5,2) 
    % Voltmetro in parallelo alla lampadina
    to[rmeter, t=V] (4.5,1)
    to[short] (3,1);
\end{circuitikz}
\end{center}

\section{Procedimento di misura}
La misura si svolge in due fasi: nella prima si prendono misure più fitte
per poter apprezzare le oscillazioni di corrente. A questo scopo è necessario
introdurre una resistenza nel circuito da utilizzare come partitore di tensione.
Inoltre, nella prima fase, ci si serve di un fotodiodo per poter captare la 
flebile luminescenza della lampadina, non visibile univocamente a occhio nudo.\\
Nella seconda fase si prendono dati meno fitti, perciò si rimuovono la 
resistenza e il fotodiodo dal circuito, non più necessari nella misura.

\section{Dati}
\begin{table}[H]
    \begin{minipage}{0.5\textwidth}

\centering
\caption*{}
\label{tab:temp2}
\begin{tabular}{|r|r|}
\hline
Tensione(V) & Corrente (A) \\ \hline
0 & 17.2 \\ \hline
30 & 17.2 \\ \hline
60 & 17.2 \\ \hline
90 & 17.2 \\ \hline
120 & 17.1 \\ \hline
150 & 17.1 \\ \hline
180 & 17.1 \\ \hline
210 & 17.2 \\ \hline
240 & 17.1 \\ \hline
270 & 17.1 \\ \hline
300 & 17.2 \\ \hline
330 & 17.2 \\ \hline
360 & 17.2 \\ \hline
390 & 36.9 \\ \hline
420 & 36.8 \\ \hline
450 & 36.7 \\ \hline
480 & 36.6 \\ \hline
510 & 36.5 \\ \hline
540 & 36.4 \\ \hline
570 & 36.4 \\ \hline
600 & 36.3 \\ \hline
630 & 36.1 \\ \hline
660 & 36.1 \\ \hline
690 & 36.0 \\ \hline
720 & 35.9 \\ \hline
750 & 35.9 \\ \hline
780 & 35.8 \\ \hline
810 & 35.8 \\ \hline
840 & 35.7 \\ \hline
870 & 35.7 \\ \hline
900 & 35.7 \\ \hline
930 & 35.7 \\ \hline
960 & 35.6 \\ \hline
990 & 35.6 \\ \hline
1020 & 35.5 \\ \hline
    \end{tabular}
    \end{minipage}
\hfill
\begin{minipage}{0.5\textwidth}


\centering
\caption*{Massa equivalente - Seconda prova}
\label{tab:temp3}

\begin{tabular}{|r|r|}
\hline
Tempo (s) & Temperatura (°C) \\ 
\hline
0 & 19.6 \\ \hline 
30 & 19.7 \\ \hline 
60 & 19.7 \\ \hline 
90 & 19.7 \\ \hline 
120 & 19.6 \\ \hline 
150 & 19.6 \\ \hline 
180 & 19.7 \\ \hline 
210 & 19.6 \\ \hline 
240 & 19.7 \\ \hline 
270 & 19.7 \\ \hline 
300 & 36.6 \\ \hline 
330 & 36.3 \\ \hline 
360 & 36.1 \\ \hline 
390 & 35.8 \\ \hline 
420 & 36.0 \\ \hline 
450 & 35.8 \\ \hline 
480 & 35.8 \\ \hline 
510 & 35.7 \\ \hline 
540 & 35.6 \\ \hline 
570 & 35.5 \\ \hline 
600 & 35.4 \\ \hline 
630 & 35.4 \\ \hline 
660 & 35.2 \\ \hline 
690 & 35.3 \\ \hline 
720 & 35.2 \\ \hline 
750 & 35.2 \\ \hline 
780 & 35.2 \\ \hline 
810 & 35.1 \\ \hline 
840 & 35.1 \\ \hline 
\end{tabular}
\end{minipage}
\end{table}
\pagebreak
\subsection{Calore specifico del metallo}
\begin{table}[H]
    \begin{minipage}{0.5\textwidth}
\centering
\caption*{metallo - Prima prova}
\begin{tabular}{|c|c|}
\hline
{Tempo (s)} & {Temperatura (°C)} \\ 
\hline
0         & 19.9 \\ \hline 
30        & 19.9 \\ \hline 
60        & 19.9 \\ \hline 
90        & 19.9 \\ \hline 
120       & 19.9 \\ \hline 
150       & 19.8 \\ \hline 
180       & 19.9 \\ \hline 
210       & 19.9 \\ \hline 
240       & 19.9 \\ \hline 
270       & 19.9 \\ \hline 
300       & 19.9 \\ \hline 
330       & 19.9 \\ \hline 
360       & 20.1 \\ \hline 
385       & 24.1 \\ \hline 
400       & 24.1 \\ \hline 
415       & 24.2 \\ \hline 
445       & 24.4 \\ \hline 
475       & 24.3 \\ \hline 
505       & 24.3 \\ \hline 
535       & 24.4 \\ \hline 
565       & 24.3 \\ \hline 
595       & 24.3 \\ \hline 
625       & 24.3 \\ \hline 
655       & 24.2 \\ \hline 
685       & 24.2 \\ \hline 
715       & 24.2 \\ \hline 
745       & 24.2 \\ \hline 
775       & 24.2 \\ \hline 
805       & 24.1 \\ \hline 
\end{tabular}
\label{tab:temperatura}
\end{minipage}
\hfill
\begin{minipage}{0.5\textwidth}
\centering
\caption*{metallo - Seconda prova}
\begin{tabular}{|c|c|}
\hline
Tempo (s)& Temperatura (°C) \\ 
\hline
0         & 20.6 \\ \hline 
30        & 20.7 \\ \hline 
60        & 20.7 \\ \hline 
90        & 20.7 \\ \hline 
120       & 20.7 \\ \hline 
150       & 20.8 \\ \hline 
180       & 20.9 \\ \hline 
210       & 20.9 \\ \hline 
240       & 20.9 \\ \hline 
270       & 20.9 \\ \hline 
300       & 20.8 \\ \hline 
330       & 24.2 \\ \hline 
350       & 24.6 \\ \hline 
360       & 24.8 \\ \hline 
370       & 25.2 \\ \hline 
400       & 25.3 \\ \hline 
430       & 25.3 \\ \hline 
460       & 25.2 \\ \hline 
490       & 25.2 \\ \hline 
520       & 25.2 \\ \hline 
550       & 25.2 \\ \hline 
580       & 25.2 \\ \hline 
610       & 25.2 \\ \hline 
640       & 25.2 \\ \hline 
670       & 25.2 \\ \hline 
700       & 25.2 \\ \hline 
730       & 25.2 \\ \hline 
760       & 25.2 \\ \hline 
790       & 25.1 \\ \hline 
820       & 25.1 \\ \hline 
850       & 25.1 \\ \hline 
880       & 25.1 \\ \hline 
910       & 25.1 \\ \hline 
940       & 25.1 \\ \hline 
970       & 25.1 \\ \hline 
\end{tabular}
\label{tab:temperatura2}
\end{minipage}
\end{table}
\pagebreak
\subsection{Calore latente di fusione del ghiaccio}
\begin{table}[H]
\footnotesize 
\begin{minipage}{.45\textwidth}
\footnotesize
\centering
\centering
\caption*{Ghiacccio - Prima prova} 
\label{tab:temp5}
\begin{tabular}{|r|r|}
\hline
Tempo (s) & Temperatura (°C) \\
\hline
0 & 19.3 \\ \hline
30 & 19.3 \\ \hline
60 & 19.3 \\ \hline
90 & 19.3 \\ \hline
120 & 19.3 \\ \hline
150 & 19.3 \\ \hline
180 & 19.3 \\ \hline
210 & 19.3 \\ \hline
240 & 19.3 \\ \hline
270 & 19.4 \\ \hline
300 & 19.4 \\ \hline
330 & 19.4 \\ \hline
360 & 19.4 \\ \hline
390 & 19.4 \\ \hline
420 & 19.4 \\ \hline
450 & 15.3 \\ \hline
455 & 14.0 \\ \hline
485 & 12.9 \\ \hline
515 & 12.2 \\ \hline
545 & 10.2 \\ \hline
575 & 10.6 \\ \hline
605 & 11.2 \\ \hline
635 & 10.6 \\ \hline
665 & 10.2 \\ \hline
695 & 9.6 \\ \hline
725 & 9.0 \\ \hline
730 & 8.5 \\ \hline
735 & 8.4 \\ \hline
740 & 8.7 \\ \hline
745 & 8.7 \\ \hline
750 & 8.7 \\ \hline
755 & 8.6 \\ \hline
760 & 8.5 \\ \hline
765 & 8.5 \\ \hline
770 & 8.4 \\ \hline
775 & 8.5 \\ \hline
780 & 8.5 \\ \hline
790 & 8.4 \\ \hline
820 & 8.5 \\ \hline
830 & 8.5 \\ \hline
860 & 8.5 \\ \hline
870 & 8.5 \\ \hline
900 & 8.5 \\ \hline
930 & 8.5 \\ \hline
960 & 8.7 \\ \hline
1030 & 8.7 \\ \hline
1060 & 8.8 \\ \hline
1090 & 8.8 \\ \hline
1120 & 8.8 \\ \hline
1150 & 8.8 \\ \hline
1180 & 8.9 \\ \hline
1210 & 8.9 \\ \hline
1240 & 9.0 \\ \hline
1270 & 9.1 \\ \hline
1300 & 9.1 \\ \hline
1330 & 9.1 \\ \hline
\end{tabular}
\end{minipage}

\label{tab:temp}
\hfill
\begin{minipage}{.45\textwidth}
        \vspace{-20.8cm}
        \centering
    \caption*{Ghiaccio - Seconda prova}
\begin{tabular}{|r|r|}
\hline
Tempo (s) & Temperatura (°C) \\
\hline
0 & 19.7 \\ \hline
30 & 19.8 \\ \hline
60 & 19.8 \\ \hline
90 & 19.8 \\ \hline
120 & 19.8 \\ \hline
150 & 19.9 \\ \hline
180 & 19.9 \\ \hline
210 & 19.9 \\ \hline
240 & 19.9 \\ \hline
270 & 19.9 \\ \hline
300 & 19.9 \\ \hline
305 & 18.3 \\ \hline
310 & 16.9 \\ \hline
315 & 14 \\ \hline
320 & 11.5 \\ \hline
325 & 11 \\ \hline
330 & 10.2 \\ \hline
335 & 10.1 \\ \hline
340 & 9.9 \\ \hline
345 & 9.7 \\ \hline
350 & 10.3 \\ \hline
355 & 7.0 \\ \hline
360 & 8.0 \\ \hline
365 & 7.3 \\ \hline
370 & 9.1 \\ \hline
375 & 9.3 \\ \hline
380 & 9.2 \\ \hline
410 & 8.3 \\ \hline
440 & 8.7 \\ \hline
470 & 8.5 \\ \hline
500 & 8.6 \\ \hline
530 & 8.7 \\ \hline
560 & 8.8 \\ \hline
590 & 8.9 \\ \hline
620 & 8.9 \\ \hline
650 & 8.9 \\ \hline
680 & 9.0 \\ \hline
710 & 9.0 \\ \hline
740 & 9.0 \\ \hline
770 & 9.2 \\ \hline
800 & 9.2 \\ \hline
830 & 9.3 \\ \hline
860 & 9.3 \\ \hline
890 & 9.3 \\ \hline
920 & 9.4 \\ \hline
950 & 9.4 \\ \hline
980 & 9.5 \\ \hline
1010 & 9.5 \\ \hline
1040 & 9.6 \\ \hline
1070 & 9.7 \\ \hline
1100 & 9.7 \\ \hline
1130 & 9.8 \\ \hline
1160 & 9.8 \\ \hline
1190 & 9.8 \\ \hline
1220 & 9.9 \\ \hline
1250 & 9.9 \\ \hline
1280 & 10 \\ \hline
1310 & 10 \\ \hline
1340 & 10.1 \\ \hline
\end{tabular}
\end{minipage}
\end{table}
\twocolumn
\section{Analisi dati}
\subsection{Massa equivalente in acqua}
L'errore su questa quantità viene calcolato utilizzando la legge di propagazione degli errori massimi:
\begin{multline*}
    \Delta m^* = \Delta m_1 + \Delta m_2 \left| \frac{T_2 - T^*}{T^* - T_1} \right| \\
    + \Delta T_2 \frac{m_2}{|T^* - T_1|} + \Delta T_1 \frac{|T_2 - T^*|}{(T^* - T_1)^2} m_2 \\
    + \Delta T^* \frac{|T_2 - T_1|}{(T^* - T_1)^2} m_2\
\end{multline*}
    dall'analisi svolta si evince che i primi 2 termini sono trascurabili:
    \begin{table}[H]
        \centering
   \begin{scriptsize}
				\noindent
		\begin{tabular}{|c|c|}
		\hline
		 termine & $\Delta$ (g)  \\ \hline
		1  & 0.01 \\ \hline
        2  & 0.01 \\ \hline
        3  & 0.20\\ \hline
        4  & 0.20 \\ \hline
        5  & 4.70 \\ \hline
		\end{tabular}
        \caption*{Massa equivalente in acqua - Seconda prova}
    \end{scriptsize}
	\end{table}
I risultati sono compatibili fra loro, perciò come dato finale si è presa la misura più precisa:
  \begin{table}[H]
    \centering
		\begin{tabular}{|c|c|}
		\hline
		$m^*$ & $\Delta$ (g)  \\ \hline
		17.0 & 5.6 \\ \hline
        19.2 & 4.3 \\ \hline
		\end{tabular}
	\end{table}
    \begin{center}
        
        \boxed{m^* = 19.2 \pm 4.3 \; g}
    \end{center}
\subsection{Calore specifico del metallo}
Si utilizza nuovamente la legge di propagazione degli errori massimi:
\begin{multline*}
\Delta m^* = \Delta m_1 + \Delta m_2 \left| \frac{T_2 - T^*}{T^* - T_1} \right| \\
+ \Delta T_2 \frac{m_2}{|T^* - T_1|} 
+ \Delta T_1 \frac{|T_2 - T^*|}{(T^* - T_1)^2} m_2 \\
+ \Delta T^* \frac{|T_2 - T_1|}{(T^* - T_1)^2} m_2
\end{multline*}
4\textsuperscript{o} e 5\textsuperscript{o} termine risultano trascurabili:
    \begin{table}[H]
        \centering
		\begin{tabular}{|c|c|}
		\hline
		 termine & errore (g)  \\ \hline
		1  & 0.1170 \\ \hline
        2  & 0.0229 \\ \hline
        3  & 0.0203\\ \hline
        4  & 0.0007 \\ \hline
        5  & 0.0005 \\ \hline
		\end{tabular}
        \caption*{Calore specifico del metallo - Seconda prova}
	\end{table}
Anche in questo caso sono state svolte due misurazioni, come in precenza si ha la compatabilità dei risultati:
  \begin{table}[H]
    \centering
		\begin{tabular}{|c|c|}
		\hline
        $c_x$ (\si{\joule\per\gram\per\kelvin}) & $\Delta c_x$ (\si{\joule\per\gram\per\kelvin}) \\ \hline
		0.56 & 0.13 \\ \hline
        0.56 & 0.16 \\ \hline
		\end{tabular}
	\end{table}

    \begin{center}
        
        \boxed{c_x = 0.56 \pm 0.13 \; \si{\joule\per\gram\per\kelvin}} 
    \end{center}
\subsection{Calore latente di fusione del ghiaccio}
Ancora una volta si è fatto uso della legge di propagazione degli errori massimi:
\begin{multline*}
\Delta \lambda_f = c_a \Bigg[
\frac{T_1 - T^*}{m_g} (\Delta m_1 + \Delta m^*) \\
+ \frac{(m_1 + m^*)(T_1 - T^*)}{m_g^2} \Delta m_g 
+ \frac{m_1 + m^*}{m_g} \Delta T_1 \\
+ \Delta T_f 
+ \frac{m_1 + m^* + m_g}{m_g} \Delta T^* 
\Bigg]
\end{multline*}
  \begin{table}[H]
    \centering
		\begin{tabular}{|c|c|}
		\hline 
        $\lambda_f$ (J/(g$\cdot$K)) & $\Delta \lambda_f$ (J/(g$\cdot$K))  \\ \hline
		327 & 45 \\ \hline
        321 & 57 \\ \hline
		\end{tabular}
	\end{table}

    \begin{center}
        
        \boxed{\lambda_f = 327 \pm 45 \; \si{\joule\per\gram\per\kelvin}} 
    \end{center}
\subsection{Temperatura d'equilibrio e Chi Squared}
\begin{nota}
Si informa il lettore che per l'analisi dei dati è stato utilizzato il software ROOT.
\end{nota}
La temperatura d'equilibrio è stata ricavata graficamente, si compie una regressione lineare con parte dei dati raccolti dal momento in cui si è raggiunto l'equilibrio termico del sistema calorimetro-sostanza. Questo 
ci permette di andare a calcolare la temperatura d'equilibrio $T^*$, nonostante gli scambi termici con l'ambiente esterno.

Volendo trovare la miglior retta che descrive il nostro set di misure e i consegueti risultati $(x_i,y_i \pm \sigma_i)$
utilizziamo la ovvia parametrizzazzione: $f_i=mx+q$, si sono dovuti quindi stimare i parametri m e q, per farlo si è utilizzato il metodo dei \textit{minimi quadrati}.
Tale principio afferma che la miglior stia dei parametri è quella che minimizza il cosidetto \textit{Chi quadrato}:
\[
\chi^2=\sum_{i=1}^{n}w_i(y_i-f(x_i))^2
\]
dove $w_i$ è peso, nel nostro caso esso diventa $\frac{1}{\sigma_i^2}$. Si devono quindi risolvere le equazioni:
$\frac{\partial X^2}{\partial m}=0$ e $\frac{\partial X^2}{\partial q}=0$, mentre gli errori si trovano a partire dalla legge di propagazione della varianza:
\begin{equation*}
\sigma_{{m}}^2 = \sum_{i=1}^N \left( \frac{\partial m}{\partial y_i} \right)^2 \sigma^2
\quad
\sigma_{q}^2 = \sum_{i=1}^N \left( \frac{\partial q}{\partial y_i} \right)^2 \sigma^2
\end{equation*}
Infine, il valore del chi quadrato per grado di libertà $\frac{\chi^2}{\text{ndf}}$, dove $\text{ndf}$ indica i gradi di libertà, nel caso del fit lineare $n - 2$ (dove $n$ è il numero di misure). Esso è un indicatore della qualità del fit.\\
Idealmente, un valore vicino a 1 indica che le incertezze sono ben stimate e il modello descrive correttamente i dati. Valori molto maggiori di 1 suggeriscono una scarsa compatibilità (oppure errori sottostimati), mentre valori molto minori di 1 possono indicare una sovrastima delle incertezze.
La \textit{probabilità del $\chi^2$}  indica quanto è plausibile ottenere il valore di $\chi^2$ osservato se il modello fosse corretto. Serve a valutare la bontà del fit: un p-value alto ($>0.05$) suggerisce che il modello descrive bene i dati, mentre uno basso indica incompatibilità.


Infine si è passati ad una trattazzione deterministica andando a valutare l'errore su ogni temperatura d'equilibrio ricavata graficamente come:
\[
\Delta T^*= \Delta m x + \Delta q
\]
Si sono ottenuti i seguenti risultati:
\begin{table}[H]
\begin{scriptsize}
\noindent
\resizebox{\columnwidth}{!}{%
\begin{tabular}{|c|c|c|c|c|c|c|c|c|c|c}
\hline
Prova &  $\sigma_m$ &  $\sigma_q$ & $\chi^2$ & $\chi^2$/ndf & $\Delta m$ & $\Delta q$ & $\Delta y$ \\
\hline
acqua 1   & 0.0001 & 0.13 & 9.2 & 1.0 & 0.0003 & 0.6 & 0.68 \\ \hline
acqua 2   & 0.0001 & 0.14 & 10.4 & 1.7 & 0.0005 & 0.4 & 0.57 \\ \hline
solido 1   & 0.0002 & 0.13 & 8.6  & 1.7 & 0.0005  & 0.4 & 0.59 \\ \hline
solido 2   & 0.0002  & 0.08 & 26.7 & 3.6 & 0.0007  & 0.2 & 0.5 \\ \hline
ghiaccio 1  & 0.0002  & 0.20 & 10.4 & 1.7 & 0.0005  & 0.5 & 0.75 \\ \hline
ghiaccio 2 & 0.00004  & 0.04 & 19.0 & 1.5 & 0.0001  & 0.12 & 0.16 \\ 
\hline
\end{tabular}}
\end{scriptsize}
\end{table}

\subsection{Tempo caratteristco}
La misura di $\tau$ diviene più semplice grazie all'analisi sopra descritta. Si utilizzano, infatti, i parametri ottenuti dal fit della retta (indicati nei grafici come $p_0$ (q) e $p_1$ (m)) sono gli stessi che si trovano in:
\[
T(t)=T_0 + \frac{T_0-T_A}{\tau}t
\]
e assumendo che il raggiugimento dell'equilibrio termico ambiente-calorimetro sia lineare si ha che:
\begin{equation*}
\tau=\frac{1}{m} \quad \Delta \tau= \frac{\Delta m}{m}\tau
\end{equation*}
da cui si ottengono i seguenti:
\begin{table}[h]
\centering
\caption*{Valori di $\tau$ e relativi errori}
\begin{tabular}{|c|c|}
\hline
$\tau$ [s] &  $\Delta \tau$ [s] \\
\hline
870 & 340 \\ \hline
560 & 37 \\ \hline
930 & 470 \\ \hline
2400 & 400 \\ \hline
730 & 180 \\ \hline
1400 & 100 \\ \hline
580 & 150 \\
\hline
\end{tabular}
\end{table}
\pagebreak
\section{Grafici}
\section{Conclusione}

\end{document}















































\end{document}